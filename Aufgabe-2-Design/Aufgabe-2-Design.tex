\documentclass{report}

% ----- IMPORTS ---------------------------------------------------------------
\usepackage[ngerman]{babel}
\usepackage{float}
\usepackage[ngerman]{datetime}
\usepackage[bookmarks]{hyperref} % PDF TOC
\usepackage{graphicx}
%\usepackage{baskervald} % Use the Baskerville font
\usepackage[utf8x]{inputenc}
\setcounter{secnumdepth}{5} % include numbering in paragraph heading

% ----- META ------------------------------------------------------------------
\title{Software Engineering\\\small{WS2015/2016}}
\author{
	\textsc{Tim Bierenbreier}\\
	\normalsize Matrikel Nr.: 43235
	\and
	\textsc{Jonas Rottmann}\\
	\normalsize Matrikel Nr.: 44501
	\and
	\textsc{Jonas Weber}\\
	\normalsize Matrikel Nr.: 43399
}

% ----- CONTENT ---------------------------------------------------------------
\begin{document}

\maketitle

\tableofcontents

\chapter{Design}

\section{Use-Cases}
\textit{Beschreiben Sie die zu realisierenden Use Cases in Form von System-Use-Cases.
\begin{itemize}
    \item Use-Case-Beschreibung
    \item Use-Case-Diagramm
    \item detaillierte Beschreibung in Form von Activity-Diagrammen
\end{itemize}
Orientieren Sie sich hierzu an Ihren Analyseergebnissen und berücksichtigen Sie die gewählte Architektur und Ihr Bedienkonzept.}

\subsection{Use-Case-Beschreibung}
    \subsubsection{Zug spielen}
    \begin{description}
        \item[Akteure]~\par
            View, Controller
        \item[Priorität]~\par
            essentiell\\Hauptbestandteil des Spiels
        \item[Beschreibung]~\par
            \begin{itemize}
                \item [1.] Würfelzähler auf 0 zurücksetzen, Spieler der am Zug ist aus Variable auslesen und vom View ausgeben lassen.
                \item [2.] Pseudozufallszahl zwischen 1 und 6 als Würfelergebnis bestimmen, den Würfelzähler inkrementieren und Wüfelergebnis vom View anzeigen lassen.
                \item [3.] Prüfen ob der Spieler am Zug schon mindestens einen Wissensstreiter auf dem Spielfeld hat.
            \end{itemize}
            \\ Fall: Spieler am Zug hat schon mindesntens einen Wissensstreiter auf dem Feld.
            \begin{itemize}
                \item [4a.] Es sind noch nicht alle Wissensstreiter auf dem Feld und das Würfelergebnis $==$ 6, falls ja muss ein Wissensstreiter aus dem Heimatfeld auf das Startfeld gebracht werden und der Zug ist beendet.
                \item [5a.] Würfelergebnis und Positionen aller Wissensstreiter von View ausgeben lassen.
                \item [6a.] View zeigt Frage welcher Wissensstreiter bewegt werden soll an und der Controller erwartet eine Eingabe.
                \item [7a.] Eingabe wird auf Gültigkeit geprüft, man kehrt ggf. zu 6a zurück.
                \item [8a.] Die Position des Wissensstreiters wird auf die alte Position + Würfelergebnis geändert, es wird geprüft ob dieses neue Feld schon besetzt ist, und ggf. der Use-Case "`Wissen testen"' gestartet. Falls dieses Feld noch nicht besetzt ist, wird dieser Spielzug beendet.
            \end{itemize}
            \\ Fall: Spieler am Zug hat noch keine eigenen Wisssensstreiter auf dem Feld.
            \begin{itemize}
                \item [4b.] Wird geprüft ob das Wüfelergebnis $==$ 6 ist, falls ja wird ein Wissensstreiter aus dem Heimatfeld auf das Startfeld gesetzt und der Zug ist beendet. Wenn dies nicht der Fall ist wird geprüft ob der Würfelzähler $<$ 3, falls ja wird zurück zu Schritt 2. gesprungen, falls nicht ist der Spielzug ebenso beendet.
            \end{itemize}
       \item[Vorbedingungen] Spieler der am Zug ist wurde in einem Feld gespeichert.
       \item[Nachbedingung] Nächster Spieler wird im Spieler am Zug Feld vermerkt.

    \subsubsection{Wissen Testen}
    \begin{description}
        \item[Akteure]~\par
            View
        \item[Priorität]~\par
            essentiell\\Hauptbestandteil des Spiels
        \item[Beschreibung]~\par

        \begin{itemize}
            \item [1.] Präsentatiosschicht fragt den prüfenden Spieler nach einer Kategorie.
            \item [2.] Applikationssschicht wählt pseudozufällig eine Frage dieser Kategorie.
            \item [3.] Präsentationsschicht zeigt die Frage und 4 Antwortmöglichkeiten für den geprüften Spieler an und fordert Eingabe.
            \item [4.] Applikationssschicht wertet Antwort aus. Ist diese falsch wird der Wissenstandsanzeiger dekrementiert (wobei der Minimalwert 0 beträgt). Ist die Antwort richtig wird geprüft ob der Wissenstandsanzeiger schon den maximal Wert erreicht hat, ist dies der Fall fragt die Präsentationsschicht welche Kategorie ansonsten erhöht werden soll.
			\item [5.] Der Wissenstreiter des geprüften Spielers wird bewegt:
				\begin{itemize}
					\item Antwort war richtig: Wissensstreiter wird auf das Startfeld des Spielers gesetzt, ist dieses bereits besetzt muss der Wissensstreiter ins Heimatfeld.
					\item Antwort war falsch: Wissensstreiter wird zurück ins Heimatfeld gesetzt.
				\end{itemize}
			\item [6.] Die Applikation vergleicht ob der prüfende Spieler $!=$ dem geprüften Spieler, falls nicht wurde der Wissenstest beendet. Stimmen prüfender Spieler und geprüfter Spieler nicht überein fragt die Präsentationsschicht den prüfenden Spieler ob er sich selbst testen will. Falls ja geht es wieder bei Schritt 1. weiter, falls nicht ist der Wissenstest beendet.
        \end{itemize}
        \item[Vorbedingungen] Wissensstreiter wurde auf Feld gezogen, welches schon besetzt ist.
    \end{description
\end{description}





\subsection{Use-Case-Diagramm}


\subsection{Aktivitäts-Diagramm}


\end{document}
