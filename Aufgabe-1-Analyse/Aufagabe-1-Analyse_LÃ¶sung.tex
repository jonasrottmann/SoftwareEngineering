\documentclass{scrartcl} %KOMA-Script replacement to <article>

% ----- IMPORTS ---------------------------------------------------------------
\usepackage[ngerman]{babel}
\usepackage[ngerman]{datetime}
\usepackage[utf8x]{inputenc}

% ----- META ------------------------------------------------------------------
\title{Software Engineering}
\subtitle{WS2015/2016} %Only in KOMA-Script documentclass
\author{
	\textsc{Tim Bierenbreier}\\
	\normalsize Matrikel Nr.: 43235
	\and
	\textsc{Jonas Rottmann}\\
	\normalsize Matrikel Nr.: 44501
	\and
	\textsc{Jonas Weber}\\
	\normalsize Matrikel Nr.: 43399
}

% ----- CONTENT ---------------------------------------------------------------
\begin{document}
\maketitle
\section{Use Cases}
\subsection{Priorisierte Use Cases}

\begin{description}
   \item[Spielfeld vorbereiten]~\par
   \begin{description}
      \item[Akteure] Spielleiter (Computer)
      \item[Priorität] essentiell\\Grundlage für den weiteren Spielverlauf.
      \item[Beschreibung] Die Spieler bestimmen die 4 Kategorien. Der Spielleiter mischt die entsprechnenden Fragen. Jeder Spieler wählt eine Farbe. Die Wissensstreiter jedes Spielers werden auf die entsprechenden Heimatfelder gesetzt und die Wissenstandsanzeiger ggf. zurückgesetzt.
      \item[Vorbedingungen] Ein Spiel wird von den Spielern gestartet.          
   \end{description}
   

   \item[Beginnenden Spieler bestimmen]~\par
   \begin{description}
      \item[Akteure] Alle Spieler
      \item[Priorität] unwichtig\\Im Entwicklungsprozess nicht wichtig, da der beginnende Spieler leicht ohne Nebenwirkungen manuell bestimmt werden kann.
      \item[Beschreibung] Alle Spieler würfeln einmal, die höchste Augenanzahl beginnt. Falls mehr als ein Spieler die höchste Zahl würfelt, müssen diese Spieler erneut gegeneinander würfeln.
      \item[Vorbedingungen] Es sind 4 Spieler bekannt und es gibt einen Würfel.
   \end{description}
   

   \item[Zug spielen]~\par
   \begin{description}
      \item[Akteure] Ein Spieler
      \item[Priorität] essentiell\\Hauptbestandteil des Spiels.
      \item[Beschreibung] Der Spieler würfelt. Wenn eine 6 fällt muss der Spieler einen Wissenstreiter auf das Spielfeld (auf das Feld seiner Farbe) bringen. Hat der Spieler keine 6 gewüfelt, oder sind bereits alle Wissensstreiter auf dem Feld, darf der Spieler einen seiner Wissensstreiter um die gewürfelte Augenzahl nach vorne ziehen. Hat der Spieler keine Wissensstreiter auf dem Feld, darf er maximal 3 mal würfeln bis eine 6 fällt. Wenn der Spielzug beendet wurde ist der Spieler zu seiner Rechten am Zug.
      \item[Vorbedingungen] Ein Spieler ist am Zug.
   \end{description}
   

   \item[Wissenstest (extends "Zug spielen")]~\par
   \begin{description}
      \item[Akteure] Ein oder zwei Spieler
      \item[Priorität] essentiell\\Hauptbestandteil des Spiels.
      \item[Beschreibung] Spieler stellt anderem Spieler Frage aus einer der 4 Kategorien.\\
Frage wird korrekt beantwortet: Wissenstandsanzeiger dieser Kategorie wird inkrementiert. Wenn der Wissenstandszeiger dieser Kategorie auf höchster Stufe ist, kann eine beliebige andere Kategorie inkrementiert werden. Der Wissensstreiter des geprüften Spielers muss auf dessen Startfeld zurückgesetzt werden, ist dieses besetzt Heimatfeld.\\
Frage konnte nicht beantwortet werden: Der Wissenstandsanzeiger dieser Kategorie wird dekrementiert. Der Wissensstreiter des geprüften Spielers kommt ins Heimatfeld.
      \item[Vorbedingungen] Spieler kommt auf ein Feld auf dem ein Wissensstreiter steht (beliebige Farbe).
   \end{description}


   \item[Anderen Spieler testen (extends "Wissenstest")]~\par
   \begin{description}
      \item[Akteure] Zwei Spieler
      \item[Priorität] essentiell\\Hauptbestandteil des Spiels.
      \item[Beschreibung] Zusätzlich: Beantwortet der zu testende Spieler die Frage falsch, kann der Fragesteller selbst eine Frage der entsprechenden Kategorie beantworten.
      \item[Vorbedingungen] Feld ist von einem fremden Wissensstreiter belegt.
   \end{description}
   

   \item[Sich selbst testen (extends "Wissenstest")]~\par
   \begin{description}
      \item[Akteure] Ein Spieler
      \item[Priorität] unwichtig\\Kann bei funktionierendem Wissenstest einfach nach implementiert werden.
      \item[Beschreibung] 
      \item[Vorbedingungen] Feld ist von einem eigenen Wissensstreiter belegt.
   \end{description}
   

   \item[Sieger bestimmen]~\par
   \begin{description}
      \item[Akteure] Spielleiter (Computer)
      \item[Priorität] unwichtig\\Für den Spielverlauf zuerst uninteressant.
      \item[Beschreibung] Zeige den Gewinner an und biete an eine neue Runde zu starten.
      \item[Vorbedingungen] Ein Spieler hat seine Wissenstandanzeige komplett gefüllt.
   \end{description}
\end{description}

\subsection{Use Case Diagram}
BILD BILD BILD

\subsection{Erste Iteration}
\begin{itemize}
   \item Spiefeld vorbereiten
   \item Zug spielen
   \item Wissenstest
\end{itemize}

\end{document}
